\section{Introduction}
More and more frequently we hear about new technologies, as robots, self-driving vehicles and drones, and their use to replace humans in some activities, especially the most repetitive or dangerous ones, \cite{art:Chui2016}, or to create infrastructures and service networks alternative to traditional ones (see for example \cite{art:Chiaraviglio2019a} and \cite{art:Amorosi2019}). In this context, many management and coordination problems arise, which can also be addressed and solved by means of optimization models. The variety of problems and applications in this area has already led to a wide scientific production and to the development of extensions of existing combinatorial optimization models (see for example \cite{art:Pugliese2017}) or to the formalization of new classes of problems, also by resorting to non-linear programming, as in \cite{art:Amorosi2021} and \cite{art:Amorosi2021b}. 
In \cite{art:Cavani2021} and \cite{art:Roberti2021,art:***} the authors study exact methods for the Traveling Salesman Problem with one or multiple drones. An interesting literature review on drone routing problems is presented in \cite{art:Macrina2020}.
In particular, the use of drone technology in various sectors is a well-studied topic that continues to receive growing interest (see, e.g. \cite{art:***,art:**, art:WANG, art:VIDAL, art:MBIA}). Indeed, in a context where the urgency of sustainable solutions with low environmental impact is growing, this technology represents a valid alternative to the use of traditional means of transport. Furthermore, drones can also reach areas that are difficult to be reached by people and in a faster and safer way. Examples of this can be found both in the parcel delivery and in many inspection and monitoring activities also in post-disaster contexts (see \cite{art:Otto2018} and \cite{art:Chung2020} for extensive surveys).
In this work, we refer to these latter activities, resorting to the use of one drone supported by a mothership vehicle working as a mobile recharging station to manage its limited endurance. Such a system requires coordination and synchronization of the vehicles involved. In \cite{art:Amorosi2021} the authors studied the Mothership Drone Routing Problem with Graphs (AMDRPG) where one drone is supported by a mothership and they formulated the coordination problem in order to visit a set of target graphs by minimizing the total distance traveled by both vehicles. In \cite{art:Amorosi2021b} an extension of this problem with multiple drones, called the Mothership and Multiple Drones Routing Problem with Graphs (MMDRPG) has been investigated. In this paper we face the case in which one drone, supported by a mothership vehicle, must visit a set of targets but, differently with respect to the previously mentioned works, in each mission, the drone can visit more than one target or a portion of it.
We assume that the mothership can move freely in the continuous space and we consider two possible shapes of the target to be visited: (i) the targets are points or (ii) the targets are polygonal chains. This is an important contribution of this work as compared with recent papers in the literature where launching and retrieving points are forced to be nodes of a given network and targets are always points \cite{art:Cavani2021,art:***, art:**}.
We mathematically formulate the problem as a Mixed Integer Non-Linear Programming (MINLP) model for which we also derive valid inequalities to reinforce it.
Moreover, in order to deal with larger size instances, we design alternative matheuristic procedures. Extensive computational experiments are performed reporting the usefulness of our exact and heuristic methods to solve the problem.

The rest of this paper is structured as follows: Section \ref{section:desc} describes the problem details and the proposed mathematical programming formulation. Section \ref{bounds} discusses a strengthening of the formulation also by resorting to valid inequalities. Section \ref{Math} presents (alternative) matheuristic procedures in order to deal with larger size instances of the problem. Section \ref{results} reports the experimental results obtained by testing the model on different instances of points and polygonal chains and the comparison with the ones provided by the matheuristic algorithms in order to evaluate their usefulness. Finally Section \ref{conclusions} concludes the paper.





\begin{comment}
In recent years the grow of the potential business opportunities related to the use of drone technology has motivated the appearance of an interesting body of methodological literature on optimizing of the use of such technology. 
\LA{We can find examples of that in many different sectors, like telecommunication where drones can be adopted in place of traditional infrastructures to provide connectivity (see for example \cite{art:Amorosi2018}, \cite{art:Chiaraviglio2018}, \cite{Jimenez2018}, \cite{art:Amorosi2019}, and \cite{art:Chiaraviglio2019a}), or to temporary deal with the damages caused by a disaster (\cite{art:Chiaraviglio2019}), deliveries (see for example \cite{art:Mathew2015} , \cite{art:Ferrandez2016}, \cite{art:Poikonen2020} and \cite{art:Amorosi2020}), also in emergency contexts (\cite{art:Wen2016}), inspection (\cite{art:Trotta2018} ) and others.
The reader is referred to the recent surveys \cite{art:Otto2018} and \cite{art:Chung2020} for further details.}\\
\noindent
Among the different aspects that can be considered we want to focus, for its relationship to the development in this paper, to the design, coordination and optimization of the combined routes of drones with a base vehicle. After the initial paper by Murray and Chu (2015) where a combined model of truck and drone is considered, Ulmer and Thomas (2018) also consider another model where trucks and drones are dispatched as order are placed and analyze the effect of different policies for either the truck or the drone. Other papers, as for instance, Campbell et al. (2017), Carlsson and Song (2017) and Dayarian et al. (2018), have also considered hybrid truck-and-drone models in order to mitigate the limited delivery range of drones. Poikonen and Golden (2019) advance on the coordination problem considering the \textit{Mothership and drone routing problem} where these two vehicles are used to design a route that visit a number of points allowing the truck to launch and recover the drone in a continuous space. More recently,  Poikonen and Golden (2020) considers the \textit{$k$-Multi-Visit drone routing problem} where a truck that acts as a mobile depot only allowed to stop in a predefined set of points, launches drones that can deliver more than one package to their designated destination points.
\noindent
Many of these papers make the assumptions that the set of allowable locations to launch/retrieve a drone are fixed and known a priori, the operations performed by the drone consist of delivering to a single point and the coordination is between a truck and a single drone. These assumptions may be appropriate in some frameworks but in other cases it may be better to relax them.\\
\noindent
\LA{
In particular, only few papers in literature focus on drones operations consisting in traversing graphs rather than visiting single points.
\cite{art:Campbell2018} introduced the \textit{Drone Rural Postman Problem} (DRPP). The authors presented a solution algorithm based on the approximation of curves in the plane by polygonal chains and that iteratively increases the number of points in the polygonal chain where the UAV can enter or leave. Thus, they solved the problem as a discrete optimization problem trying to better define the line by increasing the number of points. The authors considered also the case in which the drone has limited capacity and thus it cannot serve all the lines. To deal with this latter case, they assumed to have a fleet of drones and the problem consisted in finding a set of routes, each of limited length.\\
In \cite{art:CAMPBELL202160}  this problem has been defined as the \textit{Length Constrained K-drones Rural Postman Problem}, a continuous optimization problem where a fleet of homogeneous drones have to jointly service (traverse) a set of (curved or straight) lines of a network. The authors designed and implemented a branch-and-cut algorithm for its solution and a matheuristic algorithm capable of providing good solutions for large scale instances of the problem.\\
As regards arc routing problems involving hybrid systems consisting in one vehicle and one or multiple drones, the number of contributions in literature is further restricted.\\
In \cite{art:Tokekar2016} the authors studied the path planning problem of a system composed of a ground robot and one drone in precision agriculture and solved it by applying orienteering algorithms. \cite{art:Garone2010} studied the problem of paths planning for systems consisting in a carrier vehicle and a carried one to visit a set of target points and assuming that the carrier vehicle moves in the continuous space.\\
To the best of our knowledge, only the paper \cite{art:Amorosi2021}, deals with the coordination of a mothership with one drone to visit targets represented by graphs. The authors made different assumptions on the route followed by the mothership: it can move on a continuous framework (the Euclidean plane), ii) on a connected piecewise linear polygonal chain or iii) on a general graph. In all cases, the authors developed exact formulations resorting to mixed integer second order cone programs and proposed a matheuristic algorithm capable to obtain high quality solutions in short computing time.
\\
In this paper we deal with an extension of the problem studied in \cite{art:Amorosi2021} for which we propose a novel truck-and-multi-drones coordination model.} We consider a \LA{system} where a base vehicle (mothership) can stop anywhere in a continuous space and has to support the launch/retrieve of a number of drones that must visit graphs. The contribution on the existing literature is to extend the coordination beyond a single drone to the more cumbersome case of several drones and the operations to traversing graphs rather than visiting single points. \LA{In particular, we focus on two variants of this problem: (i) the synchronous version in which every drone is launched and retrieved in the same stage; (ii) the asynchronous version where we assume that the mothership can retrieve one drone in a different stage from the one in which it has been launched. For both variants we present a mathematical programming formulation, valid inequalities to reinforce it and a matheuristic to deal with large instances.}\\
\noindent
The work is structured as follows: Section 2 provides a detailed description of the problem under consideration and provides \LA{ mixed integer non linear programming formulations for the two variants of the problem}. Section 3  provides some valid inequalities that strengthen the formulations and also derives upper and lower bounds on the big-M constants introduced in the proposed formulations. Section 4 presents details of the matheuristic algorithms designed to handle large instances. In Section 5 we report the results obtained testing the formulation and the matheuristic algorithms on different classes of planar graphs in order to assess its effectiveness. Finally, Section 6 concludes the paper.
\end{comment}